\appendix
\setcounter{chapter}{1}
\renewcommand{\chaptermark}[1]%
	{\markboth{Appendix \thechapter. #1}{}}
\chapter{Examples of index entries}
\label{appendixB}

\section{Introduction} % a.

``Several other terms have 
been coined previously for this rapidly advancing area, such as 
\emph{connectionist processing}\index{processing!connectionist} 
\citep{shas95}, 
\emph{parallel distributed processing}\index{parallel distributed
processing}\index{PDP}
(PDP) \citep{rume86},
\emph{artificial neural networks}\index{artificial neural
network}\index{ANN!see {network, neural network}}\index{artificial neural
network|see {ANN, network, neural network}}
(ANN)~\citep{hass95,roja96},
and 
\emph{artificial neural systems}\index{artificial neural system}
\citep{zurada92}.
This book assumes a basic familiarity with neurocomputing. However, 
those readers requiring more information should page to Chapter~13 
where an introduction is given.

In \emph{knowledge-based}  neurocomputing\index{knowledge-based!neurocomputing} the emphasis is on the use 
and representation of knowledge\index{knowledge} about an application within the 
neurocomputing paradigm.''\\

This is the end of the quotation.

``Several other terms have 
been coined previously for this rapidly advancing area, such as 
\emph{connectionist processing}\index{processing!connectionist} 
\citep{shas95}, 
\emph{parallel distributed processing}\index{parallel distributed
processing}\index{PDP}
(PDP) \citep{rume86},
\emph{artificial neural networks}\index{artificial neural
network}\index{ANN!see {network, neural network}}\index{artificial neural
network|see {ANN, network, neural network}}
(ANN)~\citep{hass95,roja96},
and 
\emph{artificial neural systems}\index{artificial neural system}
\citep{zurada92}.
This book assumes a basic familiarity with neurocomputing. However, 
those readers requiring more information should page to Chapter~13 
where an introduction is given.

In \emph{knowledge-based}  neurocomputing\index{knowledge-based!neurocomputing} the emphasis is on the use 
and representation of knowledge\index{knowledge} about an application within the 
neurocomputing paradigm.''\\

This is the end of the quotation.

``Several other terms have 
been coined previously for this rapidly advancing area, such as 
\emph{connectionist processing}\index{processing!connectionist} 
\citep{shas95}, 
\emph{parallel distributed processing}\index{parallel distributed
processing}\index{PDP}
(PDP) \citep{rume86},
\emph{artificial neural networks}\index{artificial neural
network}\index{ANN!see {network, neural network}}\index{artificial neural
network|see {ANN, network, neural network}}
(ANN)~\citep{hass95,roja96},
and 
\emph{artificial neural systems}\index{artificial neural system}
\citep{zurada92}.
This book assumes a basic familiarity with neurocomputing. However, 
those readers requiring more information should page to Chapter~13 
where an introduction is given.

In \emph{knowledge-based}  neurocomputing\index{knowledge-based!neurocomputing} the emphasis is on the use 
and representation of knowledge\index{knowledge} about an application within the 
neurocomputing paradigm.''\\

This is the end of the quotation.

``Several other terms have 
been coined previously for this rapidly advancing area, such as 
\emph{connectionist processing}\index{processing!connectionist} 
\citep{shas95}, 
\emph{parallel distributed processing}\index{parallel distributed
processing}\index{PDP}
(PDP) \citep{rume86},
\emph{artificial neural networks}\index{artificial neural
network}\index{ANN!see {network, neural network}}\index{artificial neural
network|see {ANN, network, neural network}}
(ANN)~\citep{hass95,roja96},
and 
\emph{artificial neural systems}\index{artificial neural system}
\citep{zurada92}.
This book assumes a basic familiarity with neurocomputing. However, 
those readers requiring more information should page to Chapter~13 
where an introduction is given.

In \emph{knowledge-based}  neurocomputing\index{knowledge-based!neurocomputing} the emphasis is on the use 
and representation of knowledge\index{knowledge} about an application within the 
neurocomputing paradigm.''\\

This is the end of the quotation.

``Several other terms have 
been coined previously for this rapidly advancing area, such as 
\emph{connectionist processing}\index{processing!connectionist} 
\citep{shas95}, 
\emph{parallel distributed processing}\index{parallel distributed
processing}\index{PDP}
(PDP) \citep{rume86},
\emph{artificial neural networks}\index{artificial neural
network}\index{ANN!see {network, neural network}}\index{artificial neural
network|see {ANN, network, neural network}}
(ANN)~\citep{hass95,roja96},
and 
\emph{artificial neural systems}\index{artificial neural system}
\citep{zurada92}.
This book assumes a basic familiarity with neurocomputing. However, 
those readers requiring more information should page to Chapter~13 
where an introduction is given.

In \emph{knowledge-based}  neurocomputing\index{knowledge-based!neurocomputing} the emphasis is on the use 
and representation of knowledge\index{knowledge} about an application within the 
neurocomputing paradigm.''\\

This is the end of the quotation.

``Several other terms have 
been coined previously for this rapidly advancing area, such as 
\emph{connectionist processing}\index{processing!connectionist} 
\citep{shas95}, 
\emph{parallel distributed processing}\index{parallel distributed
processing}\index{PDP}
(PDP) \citep{rume86},
\emph{artificial neural networks}\index{artificial neural
network}\index{ANN!see {network, neural network}}\index{artificial neural
network|see {ANN, network, neural network}}
(ANN)~\citep{hass95,roja96},
and 
\emph{artificial neural systems}\index{artificial neural system}
\citep{zurada92}.
This book assumes a basic familiarity with neurocomputing. However, 
those readers requiring more information should page to Chapter~13 
where an introduction is given.

In \emph{knowledge-based}  neurocomputing\index{knowledge-based!neurocomputing} the emphasis is on the use 
and representation of knowledge\index{knowledge} about an application within the 
neurocomputing paradigm.''\\

This is the end of the quotation.

``Several other terms have 
been coined previously for this rapidly advancing area, such as 
\emph{connectionist processing}\index{processing!connectionist} 
\citep{shas95}, 
\emph{parallel distributed processing}\index{parallel distributed
processing}\index{PDP}
(PDP) \citep{rume86},
\emph{artificial neural networks}\index{artificial neural
network}\index{ANN!see {network, neural network}}\index{artificial neural
network|see {ANN, network, neural network}}
(ANN)~\citep{hass95,roja96},
and 
\emph{artificial neural systems}\index{artificial neural system}
\citep{zurada92}.
This book assumes a basic familiarity with neurocomputing. However, 
those readers requiring more information should page to Chapter~13 
where an introduction is given.

In \emph{knowledge-based}  neurocomputing\index{knowledge-based!neurocomputing} the emphasis is on the use 
and representation of knowledge\index{knowledge} about an application within the 
neurocomputing paradigm.''\\

This is the end of the quotation.

``Several other terms have 
been coined previously for this rapidly advancing area, such as 
\emph{connectionist processing}\index{processing!connectionist} 
\citep{shas95}, 
\emph{parallel distributed processing}\index{parallel distributed
processing}\index{PDP}
(PDP) \citep{rume86},
\emph{artificial neural networks}\index{artificial neural
network}\index{ANN!see {network, neural network}}\index{artificial neural
network|see {ANN, network, neural network}}
(ANN)~\citep{hass95,roja96},
and 
\emph{artificial neural systems}\index{artificial neural system}
\citep{zurada92}.
This book assumes a basic familiarity with neurocomputing. However, 
those readers requiring more information should page to Chapter~13 
where an introduction is given.

In \emph{knowledge-based}  neurocomputing\index{knowledge-based!neurocomputing} the emphasis is on the use 
and representation of knowledge\index{knowledge} about an application within the 
neurocomputing paradigm.''\\

This is the end of the quotation.

``Several other terms have 
been coined previously for this rapidly advancing area, such as 
\emph{connectionist processing}\index{processing!connectionist} 
\citep{shas95}, 
\emph{parallel distributed processing}\index{parallel distributed
processing}\index{PDP}
(PDP) \citep{rume86},
\emph{artificial neural networks}\index{artificial neural
network}\index{ANN!see {network, neural network}}\index{artificial neural
network|see {ANN, network, neural network}}
(ANN)~\citep{hass95,roja96},
and 
\emph{artificial neural systems}\index{artificial neural system}
\citep{zurada92}.
This book assumes a basic familiarity with neurocomputing. However, 
those readers requiring more information should page to Chapter~13 
where an introduction is given.

In \emph{knowledge-based}  neurocomputing\index{knowledge-based!neurocomputing} the emphasis is on the use 
and representation of knowledge\index{knowledge} about an application within the 
neurocomputing paradigm.''\\

This is the end of the quotation.

``Several other terms have 
been coined previously for this rapidly advancing area, such as 
\emph{connectionist processing}\index{processing!connectionist} 
\citep{shas95}, 
\emph{parallel distributed processing}\index{parallel distributed
processing}\index{PDP}
(PDP) \citep{rume86},
\emph{artificial neural networks}\index{artificial neural
network}\index{ANN!see {network, neural network}}\index{artificial neural
network|see {ANN, network, neural network}}
(ANN)~\citep{hass95,roja96},
and 
\emph{artificial neural systems}\index{artificial neural system}
\citep{zurada92}.
This book assumes a basic familiarity with neurocomputing. However, 
those readers requiring more information should page to Chapter~13 
where an introduction is given.

In \emph{knowledge-based}  neurocomputing\index{knowledge-based!neurocomputing} the emphasis is on the use 
and representation of knowledge\index{knowledge} about an application within the 
neurocomputing paradigm.''\\

This is the end of the quotation.

``Several other terms have 
been coined previously for this rapidly advancing area, such as 
\emph{connectionist processing}\index{processing!connectionist} 
\citep{shas95}, 
\emph{parallel distributed processing}\index{parallel distributed
processing}\index{PDP}
(PDP) \citep{rume86},
\emph{artificial neural networks}\index{artificial neural
network}\index{ANN!see {network, neural network}}\index{artificial neural
network|see {ANN, network, neural network}}
(ANN)~\citep{hass95,roja96},
and 
\emph{artificial neural systems}\index{artificial neural system}
\citep{zurada92}.
This book assumes a basic familiarity with neurocomputing. However, 
those readers requiring more information should page to Chapter~13 
where an introduction is given.

In \emph{knowledge-based}  neurocomputing\index{knowledge-based!neurocomputing} the emphasis is on the use 
and representation of knowledge\index{knowledge} about an application within the 
neurocomputing paradigm.''\\

This is the end of the quotation.

``Several other terms have 
been coined previously for this rapidly advancing area, such as 
\emph{connectionist processing}\index{processing!connectionist} 
\citep{shas95}, 
\emph{parallel distributed processing}\index{parallel distributed
processing}\index{PDP}
(PDP) \citep{rume86},
\emph{artificial neural networks}\index{artificial neural
network}\index{ANN!see {network, neural network}}\index{artificial neural
network|see {ANN, network, neural network}}
(ANN)~\citep{hass95,roja96},
and 
\emph{artificial neural systems}\index{artificial neural system}
\citep{zurada92}.
This book assumes a basic familiarity with neurocomputing. However, 
those readers requiring more information should page to Chapter~13 
where an introduction is given.

In \emph{knowledge-based}  neurocomputing\index{knowledge-based!neurocomputing} the emphasis is on the use 
and representation of knowledge\index{knowledge} about an application within the 
neurocomputing paradigm.''\\

This is the end of the quotation.

``Several other terms have 
been coined previously for this rapidly advancing area, such as 
\emph{connectionist processing}\index{processing!connectionist} 
\citep{shas95}, 
\emph{parallel distributed processing}\index{parallel distributed
processing}\index{PDP}
(PDP) \citep{rume86},
\emph{artificial neural networks}\index{artificial neural
network}\index{ANN!see {network, neural network}}\index{artificial neural
network|see {ANN, network, neural network}}
(ANN)~\citep{hass95,roja96},
and 
\emph{artificial neural systems}\index{artificial neural system}
\citep{zurada92}.
This book assumes a basic familiarity with neurocomputing. However, 
those readers requiring more information should page to Chapter~13 
where an introduction is given.

In \emph{knowledge-based}  neurocomputing\index{knowledge-based!neurocomputing} the emphasis is on the use 
and representation of knowledge\index{knowledge} about an application within the 
neurocomputing paradigm.''\\

This is the end of the quotation.

``Several other terms have 
been coined previously for this rapidly advancing area, such as 
\emph{connectionist processing}\index{processing!connectionist} 
\citep{shas95}, 
\emph{parallel distributed processing}\index{parallel distributed
processing}\index{PDP}
(PDP) \citep{rume86},
\emph{artificial neural networks}\index{artificial neural
network}\index{ANN!see {network, neural network}}\index{artificial neural
network|see {ANN, network, neural network}}
(ANN)~\citep{hass95,roja96},
and 
\emph{artificial neural systems}\index{artificial neural system}
\citep{zurada92}.
This book assumes a basic familiarity with neurocomputing. However, 
those readers requiring more information should page to Chapter~13 
where an introduction is given.

In \emph{knowledge-based}  neurocomputing\index{knowledge-based!neurocomputing} the emphasis is on the use 
and representation of knowledge\index{knowledge} about an application within the 
neurocomputing paradigm.''\\

This is the end of the quotation.

``Several other terms have 
been coined previously for this rapidly advancing area, such as 
\emph{connectionist processing}\index{processing!connectionist} 
\citep{shas95}, 
\emph{parallel distributed processing}\index{parallel distributed
processing}\index{PDP}
(PDP) \citep{rume86},
\emph{artificial neural networks}\index{artificial neural
network}\index{ANN!see {network, neural network}}\index{artificial neural
network|see {ANN, network, neural network}}
(ANN)~\citep{hass95,roja96},
and 
\emph{artificial neural systems}\index{artificial neural system}
\citep{zurada92}.
This book assumes a basic familiarity with neurocomputing. However, 
those readers requiring more information should page to Chapter~13 
where an introduction is given.

In \emph{knowledge-based}  neurocomputing\index{knowledge-based!neurocomputing} the emphasis is on the use 
and representation of knowledge\index{knowledge} about an application within the 
neurocomputing paradigm.''\\

This is the end of the quotation.

``Several other terms have 
been coined previously for this rapidly advancing area, such as 
\emph{connectionist processing}\index{processing!connectionist} 
\citep{shas95}, 
\emph{parallel distributed processing}\index{parallel distributed
processing}\index{PDP}
(PDP) \citep{rume86},
\emph{artificial neural networks}\index{artificial neural
network}\index{ANN!see {network, neural network}}\index{artificial neural
network|see {ANN, network, neural network}}
(ANN)~\citep{hass95,roja96},
and 
\emph{artificial neural systems}\index{artificial neural system}
\citep{zurada92}.
This book assumes a basic familiarity with neurocomputing. However, 
those readers requiring more information should page to Chapter~13 
where an introduction is given.

In \emph{knowledge-based}  neurocomputing\index{knowledge-based!neurocomputing} the emphasis is on the use 
and representation of knowledge\index{knowledge} about an application within the 
neurocomputing paradigm.''\\

This is the end of the quotation.

``Several other terms have 
been coined previously for this rapidly advancing area, such as 
\emph{connectionist processing}\index{processing!connectionist} 
\citep{shas95}, 
\emph{parallel distributed processing}\index{parallel distributed
processing}\index{PDP}
(PDP) \citep{rume86},
\emph{artificial neural networks}\index{artificial neural
network}\index{ANN!see {network, neural network}}\index{artificial neural
network|see {ANN, network, neural network}}
(ANN)~\citep{hass95,roja96},
and 
\emph{artificial neural systems}\index{artificial neural system}
\citep{zurada92}.
This book assumes a basic familiarity with neurocomputing. However, 
those readers requiring more information should page to Chapter~13 
where an introduction is given.

In \emph{knowledge-based}  neurocomputing\index{knowledge-based!neurocomputing} the emphasis is on the use 
and representation of knowledge\index{knowledge} about an application within the 
neurocomputing paradigm.''\\

This is the end of the quotation.
This is an extract from Zurada and Cloete's book to show som indexing.\\
