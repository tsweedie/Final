
\chapter{Introduction}


Communication plays a large role, in daily human interaction. It can take the form of verbal or non-verbal communication, Mehrabian found that over 93\% of verbal communication is conveyed through ones tone of voice, 38\%,  and non-verbal ques, 55\% \cite{mehrabian}. Understanding non-verbal communication is a valuable skill, in that it is a universal form of communication. 
%Verbal communication is conveyed through ones tone voice, non-verbal ques and word construction. According to Mehrrabian tone of voice and non-verbal ques account for approximately 93\%. 
Non-verbal communication is a combination of body language, physical gestures and facial expressions. Ekman \& Friesen found six facial expressions that are universally identifiable in recognizing Fear, Anger, Disgust, Surprise, Happiness and Sadness\cite{ekman}. A facial expression is made up of the changes in facial muscles mainly the mouth, eyes and eyebrows. These changes help to reflect one’s current state of mind. 

The rest of this chapter is organised as follows: Section 1.1 describes the problem statement; Section 1.2 provides the overview of the solution proposed in this project and Section 1.3 outlines the method of implementing the proposed solution.

\section{Problem Statement}
Companies use feedback from their customers as a metric to measure their customer satisfaction rate. Customer feedback can be initiated by the customer as a compliment or criticism, based on the service they were given. Alternatively, a company can offer their customers optional online or physical surveys to be completed. The problem is that customers are more likely to refrain from commenting on a service, unless provoked to do so. Customer service is a large revenue stream for companies whose core business is based around the customer. Ensuring that the customer stays happy is key, for their success. 

\section{Proposed Solution}
An Automatic Human Emotion Detection(AHED) system can be applied in any environment that benefits from understanding facial expressions and human emotion.
The proposed solution combines customer satisfaction with an automated system. This is done by using face detection to find the customers face in an image and facial feature extraction to identify the dominant customer emotion features in that image. To get the best results possible the training and classification of the system will be done using a machine learning technique. Companies can later incorporate the results of the system in improving their customer service, ensuring that their customers stay satisfied.

\section{Proposed Method}
A grayscale frontal image of the customer is used as input for the system. The Viola Jones Algorithm is then used to detect the location of the face in the image and the Histogram of Oriented Gradients(HOG)is used to extract the features. Lastly the AHED system is trained using Support Vector Machines(SVM), to classify each emotion.

