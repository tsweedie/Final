\appendix
\renewcommand{\chaptermark}[1]%
	{\markboth{Appendix \thechapter. #1}{}}
\setcounter{chapter}{3}
\chapter{Some notation}

$\overline{A}$, $A'$
\label{appendixD}

\begin{comment}

\noindent The design and linking structure of the two lessons is described here.
\section{Lesson 1}
Lesson 1 was developed in Shockwave Flash using 17 linked pages.  
In this discussion we refer to the main page as the home page and 
represent it in the diagrams with the symbol \raisebox{-3pt}[0pt][0pt]{\includegraphics{figures/home}}. 
Figure~\ref{lesson1-12-13-34} shows the links to the introduction 
and to Part 1 of Lesson 1.
\begin{figure}[ht]
\begin{center}
\includegraphics[scale=1.0]{figures/lesson1-12-13-34}
%Figure 1:  
\caption{The main links leaving from the home page and the links to and from Part 1.}\label{lesson1-12-13-34}
\end{center}
\end{figure}
Every page has a banner with five buttons.  Since Dari is written 
from the right to the left these buttons are arranged in order from 
the right to the left.  The first button returns to the home page, or 
exits from the application when the user is on the home page. 
 The other four buttons link to one of the four parts of the lesson, 
i.e.\ Part 1---set description, Part 2---the universal set and the 
empty set, Part 3---union, intersection, and difference, and Part 
4---the distributive and associative laws.  By clicking on any of 
these buttons the user is sent to the main page of the selected lesson.
Each of the four parts in our illustrations is marked similarly to 
the home page using one of the 
symbols \raisebox{-3pt}[0pt][0pt]{\includegraphics{figures/parts1234}} 
\begin{figure}[ht]
\begin{center}
\includegraphics[scale=1.0]{figures/lesson1-56}
%Figure 1:  
\caption{Part 2---the universal and empty sets}\label{lesson1-56}
\end{center}
\end{figure}
Figure~\ref{lesson1-56} shows Part 2 where set the empty set, written 
as `$\emptyset$' and the universal set, written as `\textsf{U}', and 
their relationships are discussed.  All the 
buttons in the banners function the same as usual, i.e.\ the first 
button---on the right---points to the home page, and each of the other buttons 
points to the introductory page of its part.   Each of the introductory pages 
always links to a page of comprehensive examples topically related to the 
introductory page.

Since Part 3 is more comprehensive than the other parts it seems much 
more complex than the previous two parts because it covers three operators.  The 
three set operators union, written as `$\cup$', intersection, written 
as `$\cap$', and difference, written as `/' are discussed.  The introductory 
page at the top, marked with the symbol \includegraphics{figures/three}, has three 
large pointers pointing separately to the first page of each operator.  

The operator pages function similarly to one another.  Each has a large pointer
pointing to a list of examples of the use of the operator.

Figure~\ref{lesson1-78-710-712} shows Part 3 where the lesson pages discuss the 
three set operators union, `$\cup$', intersection, `$\cap$', and difference, `/'. 
\begin{figure}[ht]
\begin{center}
\includegraphics[scale=1.0]{figures/lesson1-78-710-712}
%Figure 1:  
\caption{Part 3---union, intersection, and difference of sets}\label{lesson1-78-710-712}
\end{center}
\end{figure}
\clearpage
The back button at the bottom right-hand corner of each example page returns to its 
corresponding introductory page and the back button of the introductory page returns 
to the main page of Part 3.  The home button on all the pages returns to the main home
page.

Figure~\ref{lesson1-1415-1416-1417} shows Part 4 where the set commutative, 
associative and distributive laws are discussed.
\begin{figure}[ht]
\begin{center}
\includegraphics[scale=1.0]{figures/lesson1-1415-1416-1417}
%Figure 1:  
\caption{Part 4---the set commutative, associative, and distributive laws}\label{lesson1-1415-1416-1417}
\end{center}
\end{figure}

Figure~\ref{lesson1-16} shows the page from Part 4 that gives examples using the 
associative law in more detail.  The content of the page is fairly 
clear even though the interface is in Dari.
\begin{figure}[h]
\begin{center}
\includegraphics[scale=1.0]{figures/lesson1-16}
%Figure 1:  
\caption{Part 4---the associative law in more detail}\label{lesson1-16}
\end{center}
\end{figure}

%\clearpage

\section{Lesson 2}
Lesson 2 was developed in Visual Basic.net to incorporate some interactivity. 
 It has a similar linkage structure so very little will be gained 
by illustrating the linkage.
\end{comment}

