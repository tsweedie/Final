\chapter{Code Documentation} % Chapter 2
%

%\nocite{*}

\section{Introduction} % a.
This chapter includes all the source code used for implementing the Feature Extraction and SVM Training and Testing for the Automatic Human Emotion Detection system.
\subsection{Feature Extraction Python Code:}

    
%\begin{frame}[fragile]
 %      \begin{listing}[H]
%        \begin{pythoncode}    
\begin{lstlisting}{language=python}    
import sys
import cv2
import numpy as np
from skimage.feature import hog
from skimage import img_as_float

# a trained model for locating faces within an image
faceCascade = cv2.CascadeClassifier('Files/haarcascade_frontalface_alt.xml')

class FeatureExtraction():
    # ------------------------------------------------------------------------------
    # ----------------------------Viola-Jones Face Detection------------------------
    # ------------------------------------------------------------------------------
    def viola_jones(self, image):
        # set the height and width of face images
        height = 56
        width = 56
		
        # create an temporary 'image' of zeros
        scaled = np.zeros((height, width), dtype=np.float)
		
        # set the properties of the faceCascade
        faces = faceCascade.detectMultiScale(image, 1.3, 5)

        # set variables for finding the largest face in an image 
        max_size = 0  # w*h
        X = Y = W = H = 0
		
        # keeping track of x,y,w,h in order to find the biggest face
        for (x, y, w, h) in faces:  
            if max_size < w * h:
                max_size = w * h
                X = x
                Y = y
                W = w
                H = h
				
        # cater for when there are no faces in the image 'max_size = 0'
        if max_size != 0:
            # draw a rectangle identifying the location of the face in the image
            cv2.rectangle(image, (X, Y), (X + W, Y + H), (192, 192, 192), 2)
			
            # store the location with the face as our region of interest 'roi'
            roi = image[Y:Y + H, X:X + W]
			
            # convert the roi to grayscale
            gray = cv2.cvtColor(roi, cv2.COLOR_RGB2GRAY)
			
            # resize the roi so that all the roi's are uniform
            scaled = cv2.resize(gray, (height, width))

        return scaled, image

    # ------------------------------------------------------------------------------
    # -----------------------------------My HOG-------------------------------------
    # ------------------------------------------------------------------------------
    # Note: gradients work only on grayscale images

    # calculate the gradient in the X direction (gx)
    def gx(self, scaled):
        y, x = scaled.shape
        scaled = np.lib.pad(scaled, 1, 'constant', constant_values=0)
        gx = np.zeros((x, y))
        # sobelX:
        # [1	,0	,-1]
        # [2	,0	,-2]
        # [1	,0	,-1]
        for i in range(y):
            a = np.convolve(scaled[i - 1, :], [1, 0, -1], 'valid')
            b = np.convolve(scaled[i, :], [2, 0, -2], 'valid')
            c = np.convolve(scaled[i + 1, :], [1, 0, -1], 'valid')
            gx[i, :] = np.sum([a, b, c], axis=0)
        return gx
		
    # calculate the gradient in the Y direction (gy)
    def gy(self, scaled):
        y, x = scaled.shape
        scaled = np.lib.pad(scaled, 1, 'constant', constant_values=0)
        gy = np.zeros((x, y))
        # sobelY:
        # [-1	,-2	,-1]
        # [0	,0	,0]
        # [1	,2	,1]
        for j in range(x):
            a = np.convolve(scaled[:, j - 1], [1, 0, -1], 'valid')
            b = np.convolve(scaled[:, j], [2, 0, -2], 'valid')
            c = np.convolve(scaled[:, j + 1], [1, 0, -1], 'valid')
            gy[:, j] = np.sum([a, b, c], axis=0)
        return gy

    # calculate magnitude of the gradient('intensity')
    def magnitude(self, gx, gy):
        magnitude = np.sqrt(gx ** 2 + gy ** 2)
        return magnitude

    # calculate orientation of the gradient('direction')
    def orientation(self, gx, gy):
        # output is the same output as cv2.cartToPolar
        orientation = (np.arctan2(gy, gx) / np.pi) % 2 * np.pi  
        return orientation
	
    # calculate the HOG using an algorithm developed from my documentation
    def calculate_myhog(self, scaled):
        # gradients gx and gy
        gx = self.gx(scaled)
        gy = self.gy(scaled)

        # magnitude and orientation
        magnitude = self.magnitude(gx, gy)
        orientation = self.orientation(gx, gy)

        # orientation bins
        bin_n = 9
        bins = np.int32(bin_n * orientation / (2 * np.pi))

        # magnitude and orientation cells within a block
        bin_blocks = []
        mag_blocks = []
        epsilon = sys.float_info.epsilon

        # size of block
        blocksize = 3 
		
        # size of cell
        cellsize = 4

        # store the height and width of the face image(roi)
        width = scaled.shape[0]
        height = scaled.shape[1]

        # create the parameters for the block slider
        y = ((height - (cellsize * blocksize)) / cellsize) + 1
        x = ((width - (cellsize * blocksize)) / cellsize) + 1

        # stores all the histograms in an image
        histograms = []
		
        # loops through each block in a image(i,j)
        # using a block "slider" to capture all posible blocks 
        for i in range(0, y, 1):  
            for j in range(0, x, 1):
                # magnitude and orientation cells within a block
                bin_block = bins[i * cellsize: i * cellsize + blocksize * cellsize,
                            j * cellsize: j * cellsize + blocksize * cellsize]
                mag_block = magnitude[i * cellsize: i * cellsize + blocksize * cellsize,
                            j * cellsize: j * cellsize + blocksize * cellsize]
							
                tempHists = []
                sumHists = np.zeros((9,))
				
                # loops through each cell in a block(m,n)
                # this ensures that all the histograms in each cell are calculated
                for m in range(0, blocksize * cellsize, cellsize): 
                    for n in range(0, blocksize * cellsize, cellsize):
					
                        # magnitude and orientation cells within a cell
                        cellBin = bin_block[m:m + cellsize, n:n + cellsize]
                        cellMag = mag_block[m:m + cellsize, n:n + cellsize]
						
                        # temporarily store the histograms
                        tempHists.append(np.bincount(cellBin.ravel(), 
                        cellMag.ravel(), bin_n))
						
                        # store the sum of the histograms within a block
                        # this will be used for block normalization
                        sumHists = np.sum([sumHists, np.bincount(cellBin.ravel(), 
                        cellMag.ravel(), bin_n)], axis=0)
						
                # sum up all the bins
                sumHists = np.sum(sumHists)
				
                # normalize all the cell histograms in the block
                for hist in tempHists:
                    histograms.append(np.divide(hist, np.sqrt(
                    np.add(np.square(sumHists), np.square(epsilon)))))

                # store the magnitude and orientation for the block
                bin_blocks.append(bin_block)
                mag_blocks.append(mag_block)
        # create the HOG feature vector
        hist = np.hstack(histograms)
        return hist

    # ------------------------------------------------------------------------------
    # ---------------------------------OpenCV HOG-----------------------------------
    # ------------------------------------------------------------------------------
	
    # HOG implementation from skimage, using prefered parameters
    def hog_opencv(self, image):
        image = img_as_float(image)  # convert unit8 tofloat64 ... dtype
        orientations = 9		# orientation bins
        cellSize = (4, 4)		# pixels_per_cell
        blockSize = (3, 3)		# cells_per_block
        blockNorm = 'L1-sqrt'	# {'L1', 'L1-sqrt', 'L2', 'L2-Hys'}
        visualize = True		# Also return an image of the HOG.
        transformSqrt = False
        featureVector = True
        fd, hog_image = hog(image, orientations, cellSize, 
        blockSize, blockNorm, visualize, transformSqrt,
                            featureVector)
        return fd
		
    # testing features 
    def main(self):
        image = cv2.imread('Files/lense.png')
		scaled, image = self.viola_jones(image)
        print(self.calculate_myhog(scaled).shape)
		fd = self.hog_opencv(scaled)
		print(fd.shape)


if __name__ == '__main__': FeatureExtraction().main()
\end{lstlisting}%       \end{pythoncode}
%        \caption{Viola Jones Face Detection and HOG Feature Extraction, including self implemented HOG}
%    \end{listing}
%\end{frame}

 \clearpage
\subsection{SVM Training and Testing Python Code:}
%\begin{frame}[fragile]
      %\begin{listing}[H]
%        \begin{pythoncode}
\begin{lstlisting}{language=python}    
import sys
import pandas as pd
from sklearn import svm
from sklearn.model_selection import GridSearchCV, cross_val_score, StratifiedKFold, 
     train_test_split
from sklearn.metrics import confusion_matrix, accuracy_score, classification_report
import pickle


class Tester():
    def run_test(self, data, modelname):
        print '\n######################################################################'
        print '##########################~', modelname, '~#####################'
        print '######################################################################'

        dataset = pd.read_csv(data)

        X = dataset.ix[:, 1:].values
        y = dataset.ix[:, 0].values

        X_train, X_test, y_train, y_test = train_test_split(X, y, 
        test_size=.40, stratify=y, random_state=40)

        # Grid search parameters:
		
        # C = np.logspace(-5, 15,num=21,base = 2.0)
        # gamma = np.logspace(-15, 3, num=19,base = 2.0)
        param_grid = [
            {'C': [1, 10, 100, 1000], 'kernel': ['linear']},
            {'C': [1, 10, 100, 1000], 'gamma': [0.001, 0.0001], 'kernel': ['rbf']},
        ]
		
		# choose svm type: SVC - Support Vector Classification(based on libsvm)
        svc = svm.SVC(probability=True)

        print '\n######################################################################'
        print '#############################~SVM Grid Search~########################'
        print '######################################################################'
        clf = GridSearchCV(svc, param_grid)
        print '\t Parameter Grid:\n', param_grid

        print '\n######################################################################'
        print '##########################~SVM Cross Validation~######################'
        print '######################################################################'
        skf = StratifiedKFold(n_splits=3, random_state=None, shuffle=False)
        scores = cross_val_score(clf, X, y, cv=skf, n_jobs=-1)
        print '\t Cross validation scores: \t', scores
        print '\t Cross validation Accuracy:
        		\t %0.2f (+/- %0.2f)' % (scores.mean(), scores.std() * 2)

        print '\n######################################################################'
        print '#############################~SVM Train~##############################'
        print '######################################################################'
        clf = clf.fit(X_train, y_train)
        filename = modelname
        pickle.dump(clf, open('Files/'+filename, 'wb'))
        print '\t Best score for classifier:\t', clf.best_score_
        print '\t Best C:\t', clf.best_estimator_.C
        print '\t Best Kernel:\t', clf.best_estimator_.kernel
        print '\t Best Gamma:\t', clf.best_estimator_.gamma
        print '\t SVM Best Estimator:\t', clf.best_estimator_
        print '\n\t SVM Grid Scores: \n', clf.cv_results_

        print '\n######################################################################'
        print '#############################~SVM Predict~############################'
        print '######################################################################'
        clf = pickle.load(open('Files/'+modelname, 'rb'))
        y_pred = clf.predict(X_test)
        print 'SVM Classification Report:'
        print classification_report(y_test, y_pred,
                                    target_names=['Angry','Disgust','Fear',
                                    'Happy','Neutral','Sad', 'Surprise'])
        print 'SVM Confusion Matrix:'

        print confusion_matrix(y_test, y_pred, labels = [1,2,3,4,5,6,7])
        print '\t SVM Accuracy Score:', accuracy_score(y_test,y_pred,normalize=True)
        print 'file:', data

        print '\n######################################################################'
        print '#############################~SVM Test Results~############################'
        print '######################################################################'
        for i in range(len(X_test_names)):
            print 'Subject:',X_test_names[i],'\t Label:',y_test[i],
            		'\t Prediction:',y_pred[i]

    def main(self):
        print 'Output is printed to Files/test.out'
        orig_stdout = sys.stdout
        output = open('Files/test.out', 'w+')
        sys.stdout = output
        self.run_test('Files/dataCsv.csv', 'finalized_model.sav')
        sys.stdout.close()
        sys.stdout = orig_stdout
        print 'Done ^_^'
if __name__ == '__main__': Tester().main()
\end{lstlisting}%       \end{pythoncode}
%        \caption{SVM Training, Testing and Model Selection \& Optimization}
%    \end{listing}
%\end{frame}
