\chapter{Testing} % Chapter 2
%


%\nocite{*}

\section{Introduction} % a.
This chapter looks at the results from training, testing and optimizing of the SVM Model selection process, in Section~\ref{sec:svm}. Table \ref{table: terms} gives an overview of the formulas and terms used in Figure ~\ref{fig:conf} and Table \ref{table:class} to describe the results.
\begin{table}[H]
\centering
\resizebox{\textwidth}{!}{
\begin{tabular}{ |c||c|c|}
	\hline
	\multicolumn{3}{|c|}{\textbf{SVM Model Evaluation}}\\
	\hline
      	\textbf{Term} &  \textbf{Formula} & \textbf{Description}\\
	\hline
      	\textbf{Type I Error} &      FP  &    False Positive   \\
	\hline
   	\textbf{Type II Error} &     FN  &    False Negative \\
	\hline
       	\textbf{Accuracy} &  $\frac{TP+TN}{TP+TN+FP+FN}$ & Evaluates the degree of correctness for the predictions   \\
	\hline
      	\textbf{Precision} &  $\frac{TP}{TP+FP}$ 	 &   The Positive predictive value \\
	\hline
    	\textbf{Recall} &     $\frac{TP}{TP+FP}$	 &   True positive rate  \\
	\hline
        \textbf{F1-Score} &  $2\times\frac{precision \times recall}{precision+recall}$  &   Evaluates the accuracy of predictions \\
	\hline
\end{tabular}}
\caption{Terminology and formulas used for evaluating the SVM Model\cite{dict}}
\label{table: terms}
\end{table}

%The testing of the entire system is followed in Section~\ref{sec:ahed}.

\subsection{Analysis of SVM Testing Results}\label{sec:svm}
The test set for testing the svm model consisted of 136 entries, which is 40\% of the initial CK+ dataset. The overall accuracy of the SVM Model is 88.2\%, across all subjects and emotions. Table \ref{table:class} contains the 'SVM Model Classification Report'. The report indicates the performance of each individual class, or emotion, and the overall estimated performance of the SVM model with regards to the precision, recall and f1-score. Classes that had more data available performed better overall as compared to those that had less. Since there was more testing data available for these classes. Working with an uneven dataset makes it harder to judge the performance of each class in comparison to the other classes. 
The emotions labelled Disgust, Happy and Surprise had over 24 subjects in the test set. This resulted in a higher f1-score for prediction accuracy of these emotions. Where the emotions labelled Sad, Neutral, Fear and Angry had significantly lower f1-score and fewer subjects in the test set. 
\begin{table}[H]
\centering
\resizebox{\textwidth}{!}{
\begin{tabular}{ |c||c|c|c|c|}
	\hline
	\multicolumn{5}{|c|}{\textbf{SVM Model Classification Report}}\\
	\hline
      	\textbf{Label} &      \textbf{precision} &   \textbf{recall} & \textbf{f1-score} &  \textbf{support}\\
	\hline
      	\textbf{Angry} &      0.74  &    0.82  &    0.78  &      17\\
   	\textbf{Disgust} &      1.00  &    0.92  &    0.96  &      24\\
       	\textbf{Fear} &      0.86  &    0.60  &    0.71  &      10\\
      	\textbf{Happy} &      0.93  &    1.00  &    0.96  &      27\\
    	\textbf{Neutral} &      0.67  &    0.71  &    0.69  &      14\\
        \textbf{Sad} &      0.75  &    0.82  &    0.78  &      11 \\
   	\textbf{Surprise} &      1.00  &    0.97  &    0.98  &      33 \\
	\hline
	\textbf{Avg \/ Total}  &     0.89 &     0.88  &   0.88    &   136\\
	\hline
\end{tabular}}
\caption{SVM Classification Report}
\label{table:class}
\end{table}
\subsection{Analysis of Confusion Matrix for Test Results}
\begin{figure}[H]
  \centering
  \includegraphics[scale=1.5]{conf1}
  \caption{Confusion Matrix of SVM Model Classification}
  \label{fig: res1}
\end{figure} 


\subsection{Analysis of SVM Testing Results for Subjects}
The results in the table,In Figure ~\ref{fig: res1} and ~\ref{fig: res2}, look at all the individual subjects within the test set and their classification performance using the svm model. The original CK+ dataset is uneven in that not all subjects had all seven emotions present in the dataset. This made it challenging to split the dataset evenly based on subjects. The spilt was done by ensuring that each emotion had the same test-train ratio for the training dataset and the testing dataset. In the results table, the green represents a correct classification under the indicated "Emotion Label" and the red blocks indicate a misclassification for that emotion and the misclassification is included in white text. Each subject had at least one emotion classified during testing, with a maximum of five emotions for subject 'S055'. 

\begin{figure}[H]
  \centering
  \includegraphics[scale=1.5]{res1}
  \caption{SVM Testing and Training Results for Subjects}
  \label{fig: res1}
\end{figure} 
\begin{figure}[H]
  \centering
  \includegraphics[scale=1.5]{res2}
  \caption{SVM Testing and Training Results for Subjects}
  \label{fig: res2}
\end{figure} 


%\clearpage

%\subsection{Testing of AHED System}\label{sec:ahed}





