
\chapter{Introduction}


Communication plays a large role, in daily human interaction. It can take the form of verbal or non-verbal communication, Mehrabian found that over 93\% of verbal communication is conveyed through, the tone of the voice (38\%) and non-verbal ques(55\%)\cite{mehrabian}. Understanding non-verbal communication is a valuable skill, in that it is a universal form of communication. 
%Verbal communication is conveyed through tone voice and non-verbal quotes and words. According to Mehrrabian tone of voice and non-verbal quotes account for approximately 93\%. 
Non-verbal communication is a combination of body language, physical gestures and facial expressions. Ekman \& Friesen found six facial expressions that are universally identifiable in recognizing Fear, Anger, Disgust, Surprise, Happiness and Sadness\cite{ekman}. A facial expression is made up of the changes in facial muscles (mouth, eyes, eyebrows etc.), these changes reflect one’s current state of mind. 

\section{Problem Statement}
Companies use feedback from their customers as a metric to measure their customer satisfaction rate. Customer feedback can be initiated by the customer as a compliment or criticism, based on the service they were given. Alternatively, a company can offer their customers optional online or physical surveys to be completed. The problem is that customers are more likely to refrain from commenting on a service, unless provoked to do so.

\section{Proposed Solution}
Customer service is a large revenue stream for companies whose core business is based around the customer. Ensuring that the customer stays happy is key, for their success. An Automatic Human Emotion Detection (AHED) system can be applied in any environment that can benefit from understanding facial expressions.
The proposed solution combines customer satisfaction with an automated system. Using face detection and facial feature extraction to identify the dominant customer emotion. The training and classification of the system will be done using a machine learning technique. Companies can incorporate the results of the system in improving their customer service, ensuring that their customers stay satisfied.

\section{Proposed Method}

The proposed method for this project, first captures an image from the camera, the image is processed using the Viola Jones Algorithm to detect the face and the Histogram of Oriented Gradients(HOG) to extract the features from the face. Lastly the AHED system is trained using Artificial Neural Networks, to classify each emotion. 


%Customer service is a large revenue stream for some companies, therefore ensuring that they provide the best quality service is likely to be their main priority. Not all customers readily express their emotions, verbally, regarding the quality of the service they are provided. Mehrabian 1980, states that 55\% of communication is facial expression. The motivation for this project is to apply an Automatic Human Emotion Detection(AHED) system in cases where an employee interacts with a customer. The AHED system focuses on emotion recognition using facial expressions. The proposed method for this project, first captures an image from the camera, the image is processed using the Viola Jones Algorithm to detect the face and the Pyramid Histogram of Oriented Gradients(PHOG) to extract the features from the face. Lastly the AHED system is trained using Artificial Neural Networks, to classify each emotion. There are six universal human expressions described by Ekman and Friesen, namely Surprise, Fear, Disgust, Anger, Happiness and Sadness. Grayscale frontal images will be used as input for the AHED system.
